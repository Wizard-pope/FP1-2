\documentclass[a4paper]{report}
\author{Jure Kos}
\title{Vaja 63, Meritve spektra z uklonsko mrežico}
\date{15.4.2022}
\usepackage{graphicx}
\usepackage{gensymb}
\graphicspath{ {./images/} }

\begin{document}


\maketitle

\chapter*{Uvod}
Uklonska mrežica je ploščica, v kateri so enakomerno, na gosto zarezane tanke črte. Razmik med zaporednima zarezama je mrežna konstanta.

Skozi mrežico pravokotno posvetimo z vzporednim curkom svetloba. Svetloba se pri prehodu skozi reže, uklanja. Curki uklonjene svetlobe med seboj interferirajo. Ko je sledeči pogoj izpolnjen, se curki ojačajo v smereh:

\[d  \sin \alpha = n  \lambda\]

$d$ je mrežna konstanta, $\alpha$ pa kot med smerjo vpadajočega in uklonjenega curka, $n$ pa iznačuje red curka. Najvišji red je določen z razmerjem $d / \lambda$

Če mrežico zasučemo tako, da oklepa določen kot $\varphi$ glede na pravokotnico, se uklonjeni curki premaknejo. Dobimo jih v smereh $\vartheta$ in $\vartheta ' $ glede na vpadli curek, v katerih velja

\[d [\sin(\vartheta' + \varphi) - \sin \varphi] = n\lambda\]

\[d [\sin(\vartheta - \varphi) + \sin \varphi] = n\lambda\]

Če svetimo v uklonsko mrežico s svetlobo, katere izvir je razredčen plin (živosrebrne pare), ne dobimo zveznega, temveč črtast spekter ojačitev in  oslabitev. 



\chapter*{Naloga}

S spektroskopom na uklonsko mrežico izmeri spekter živosrebrne pare.

\section*{Potrebščine}
1. Živosrebrna svetilka z dušilko\\
2. uklonska mrežica z mrežno konstanto ($d = \frac{1}{600}mm$)\\
3. spektroskop

\chapter*{Meritve in obdelava podatkov}

\section*{Uklonska mrežica v pravokotni legi}

Prvi in drugi red ojačitve pri $\varphi = 0 \degree$, merjeno v levo:


\begin{center}
  \begin{tabular}{|c|c|}
  \hline
    $\alpha [\degree] \pm 0.1 \degree$ & $\lambda [nm] \pm 50nm$\\ 
    \hline
    12,5 &   361\\
    16,2 &   465 \\
    17,3 &   496\\
    17,4 &   498 \\ \hline
    19,3 &   275\\
    23,2 &   328\\
    24,4 &   344\\
    24,5 &   346 \\ \hline
  \end{tabular}
\end{center}

Prvi in drugi red ojačitve pri $\varphi = 0 \degree$, merjeno v desno


\begin{center}
  \begin{tabular}{|c|c|} 
  \hline
    $\alpha [\degree] \pm 0.1\degree $& $\lambda [nm] \pm 50nm$\\ 
    \hline
    11,0 &   318\\
    15,0 &   431\\
    16,1 &   462\\
    16,2 &   465\\\hline
    18,1 &   259\\
    22,3 &   316\\
    23,4 &   331\\
    23,5 &   332 \\ \hline
  \end{tabular}
\end{center}


\newpage
\section*{Uklonska mrežica v zasukana za kot $\varphi$}

Prvi, drugi in tretji red ojačitve pri $\varphi = 20 \degree$, merjeno v levo:

\begin{center}
  \begin{tabular}{|c|c|}
  \hline
      $\alpha [\degree] \pm 0.1\degree $& $\lambda [nm] \pm 50nm$\\ 
      \hline
      19,2 &   483\\
      22,9 &   472\\
      24,0 &   506\\
      24,1 &   508\\ \hline
      34,1 &   390\\
      35,6 &   403\\
      / &  /\\
      / &  /\\ \hline
      / &  /\\
      41,8 & 299 \\
      44,1 & 309 \\
      44,3 & 310 \\ \hline
  \end{tabular}
\end{center}

Prvi red ojačitve pri $\varphi = 20 \degree$, merjeno v desno:


\begin{center}
  \begin{tabular}{|c|c|} \hline
      $\alpha [\degree] \pm 0.1\degree$ & $\lambda [nm] \pm 50nm$\\ 
      \hline
      13,5 &   349\\
      18,7 &   472\\
      20,2 &   506\\
      20,3 &   508\\ \hline
  \end{tabular}
\end{center}

\end{document}