\documentclass[a4paper]{report}
\author{Jure Kos}
\title{Vaja 45, Tuljava v magnetnem polju}
\date{3.3.2022}
\usepackage{graphicx}
\graphicspath{ {./images/} }

\begin{document}

\maketitle

\chapter*{Uvod}

Magnetni dipolni moment je lastnost mnogih teles, med drugim permanentnih magnetov, tokovnih zank pa tudi elektronov in atomov. V zunanjem magnetnem polju B na telo z magnetnim dipolnim momentom $p_m$ deluje navor.

\[M = P_m \times B\]

\noindent Za tuljavo s presekom S in z N ovoji, skozi katero teče tok I, velja

\[P_m = NIS\]

\noindent Smer površinskega vektorja S pove gibanje desnega vijaka, ki ga sukamo v smeri
toka. Navor na tuljavo v zunanjem magnetnem polju je tako enak

\[M = NIS \times B\]

\noindent Pri vaji bomo kot vir zunanjega magnetnega polja uporabili par Helmholtzovih
tuljav. To je priprava, ki jo sestavljata dve enaki okrogli zaporedno vezani tuljavi
(vsaka z $N_H$ ovoji), ki sta nameščeni na isti osi in sta med seboj oddaljeni toliko,
kot znaša njun radij $R_H$. Kadar skozi njiju teče električni tok $I_H$ v isti
smeri, kaže gostota magnetnega polja okoli centra postavitve v smeri osi in je precej
homogena. Njeno velikost lahko izpeljemo iz Biot-Savartovega zakona, dobimo:

\[B = \left( \frac{4}{5}\right)^{3/2} \frac{\mu_0N_HI_H}{R_H}\]

\noindent Velikost navora je tako

\[M = \left( \frac{4}{5}\right)^{3/2} \frac{\mu_0NN_HI_HIS}{R_H} sin\varphi\]

\noindent kjer je $\varphi$ kot med vektorjema S in B.

\section*{Naloga}

Z uravnovešenjem navora na tuljavo v homogenem magnetnem polju Helmholtzove
tuljave določiti indukcijsko konstanto.

\section*{Potrebščine}

1. Par Helmholtzovih tuljav s polmerom $R_H$=200 mm, vsaka $N_H$=154 ovojev,\\
2. merilna tuljava (na voljo različne),\\
3. stojalo za merilno tuljavo s torzijskim merilcem navora,\\
4. tokovni usmernik za Helholtzovi tuljavi,\\
5. tokovni usmernik za merilno tuljavo.

\chapter*{Meritve}
\section*{Dimenzije tuljav}
Mala:\\
N=3\\
2R=11,8cm$\pm$0,5mm\\
r=10cm$\pm0,1cm$\\
\\
Velika:\\
N=154\\
2R=40cm

\section*{Meritve sil}
\begin{center}
\begin{tabular}{ |c|c|c| } 
 \hline
 $F[mN]$ & $I[mA]$ & $I_H[A]$\\
 \hline \hline 
 0,15  & 500   & 1 \\
    0,20  & 750   & 1 \\
    0,25  & 1000  & 1 \\
    0,30  & 1250  & 1 \\
    0,35  & 1500  & 1 \\
    0,40  & 1750  & 1 \\
    0,45  & 2000  & 1 \\
    0,50  & 2250  & 1 \\
    0,55  & 2500  & 1 \\
    0,60  & 2750  & 1 \\
    \hline
    0,65  & 500   & 1,5 \\
    0,20  & 750   & 1,5 \\
    0,20  & 1000  & 1,5 \\
    0,40  & 1250  & 1,5 \\
    0,50  & 1500  & 1,5 \\
    0,60  & 1750  & 1,5 \\
    0,65  & 2000  & 1,5 \\
    0,70  & 2250  & 1,5 \\
    0,80  & 2500  & 1,5 \\
    0,90  & 2750  & 1,5 \\
    \hline
    0,25  & 500   & 2 \\
    0,35  & 750   & 2 \\
    0,45  & 1000  & 2 \\
    0,55  & 1250  & 2 \\
    0,65  & 1500  & 2 \\
    0,75  & 1750  & 2 \\
    0,90  & 2000  & 2 \\
    1,00  & 2250  & 2 \\
    1,10  & 2500  & 2 \\
    1,20  & 2750  & 2 \\
    \hline
    0,25  & 500   & 2,5 \\
    0,40  & 750   & 2,5 \\
    0,65  & 1000  & 2,5 \\
    0,70  & 1250  & 2,5 \\
    0,85  & 1500  & 2,5 \\
    1,00  & 1750  & 2,5 \\
    1,20  & 2000  & 2,5 \\
    1,25  & 2250  & 2,5 \\
    1,35  & 2500  & 2,5 \\
    1,50  & 2750  & 2,5 \\
    
 \hline
\end{tabular}
\end{center}
\pagebreak

\section*{Računi}

Indukcijsko konstanto lahko izračunamo po enačbi

\[\mu_0 =\left(\frac{5}{4}\right)^{3/2}\frac{MR_H}{NN_HI_HISsin\varphi}\]

\noindent Za vse meritve na koncu dobimo dobimo $\mu_0$ kot

\[\mu_0=1,0\cdot10^{-6}Vs/Am \pm 0,3\cdot10^{-6}Vs/Am \]

\section*{Vprašanja}
1. Smer magnetnega polja določimo s pravilom desnega vijaka.\\
\\
\noindent 2. Vektor navora je pravokoten na magnetno polje.

\end{document}