\documentclass[a4paper]{report}
\usepackage[utf8]{inputenc}
\usepackage[utf8]{inputenc}
\usepackage[utf8]{inputenc}
\usepackage[utf8]{inputenc}
\usepackage[slovene]{babel}
\usepackage[euler]{textgreek}
\usepackage[T1]{fontenc}
\usepackage{xcolor}
\usepackage{cite}
\usepackage{graphicx}
\usepackage{subcaption}
\usepackage{natbib}
\usepackage{hyperref}
\usepackage{multicol}
\usepackage{float}
\usepackage{amssymb,amsfonts,amsmath}
\usepackage{dsfont}
\usepackage{authblk}
\usepackage{caption}
\usepackage{ragged2e}
\usepackage{blindtext}
\DeclareGraphicsExtensions{{.jpg},{.png},{.pdf}}
%\doublespacing
%\usepackage{lipsum}
\usepackage{natbib}
\usepackage{graphicx}
\begin{document}
\title{Vaja 32 Sklopljeno nihanje}
\author{Jure Kos }
\date {5.1.2022}
\maketitle


\chapter*{Uvod}
Oglejmo si nihanje nihala, sestavljenega iz dveh enakih težnih nihal, povezanih s prožno vzmetjo. Če vzmet odstranimo, niha vsako nihalo zase s frekvenco $$
\omega_0 = \sqrt{D/J}$$, torej niha z
nihajnim časom $t_0 = 2\pi \sqrt{J/D}$, kjer je J vztrajnostni moment nihala in D koeficient
navora. Ko obe nihali povežemo z vzmetjo, ne moreta več nihati neodvisno, ampak
vplivata drugo na drugo. Pravimo, da sta nihali sklopljeni. Račun pokaže, da
lahko poljubno nihanje dveh sklopljenih nihal opišemo z linearno kombinacijo dveh
sinusnih nihanj, ki jih imenujemo lastni nihanji. Frekvenci lastnih nihanj sta lastni
frekvenci, nihajna časa pa lastna nihajna časa.
Pri tej vaji želimo izračunati gravitacijski pospešek Zemlje. Pri tem uporabimo popravke, ki nam zgornjo formulo naredijo natančnejšo.
\\


\chapter*{Meritve}
Da bi najprej dobili koeficient vzmeti, smo ob spreminjanju sile opazovali raztezek vzmeti. Podatki so zbrani v tabeli\\
\begin{table}[H]
    \centering
    \begin{tabular}{c|c|c}
         &x [m]&F [N]  \\
         \hline
 1&        0.011&	0.2424\\
2&0.023&	0.4996\\
3&0.06&	1.4774\\
    \end{tabular}
    \caption{Tabela meritev za izračun koeficienta vzmeti}
    \label{tab:my_label}
\end{table}
Koeficient vzmeti določimo iz linearne regresije Hookovega zakona, saj velja $F=kx$. koefeicent $k$ je torej strmina premice na grafu F(x). S programom \textit{gnuplot} sem prilagodila premico na meritve in dobila funkcijo
$$F(x) = (24,1885 \pm 0,7224) \frac{N}{m} \cdot x,
$$ oziroma koeficient vzmeti je $ k = 24,1885 \pm 0,7224 \frac{N}{m} = 24,1885 (1 \pm 0,02995) \frac{N}{m} $
 \begin{figure}[H]
     \centering
     %\includegraphics[width=\textwidth]{Graf 32 koeficient.png}
     \caption{Graf meritev in prilagojena premica}
     \label{fig:my_label}
 \end{figure}
\section*{Odklon nihal v isti smeri}
Pri prvem delu smo obe nihali odklinili v isto smer in opazovali gibanje.\\
Izmerili smo sledeče podatke:\\
\begin{itemize}
    \item amplituda $x_0 = 10 \ cm \pm 0,5 \ cm$
    \item dolžina nihala $d_0= 85,5 \ cm \pm 0,1 \ cm$
    \item masa nihala $m=1270 \ g \pm 1\ g$
\end{itemize}


Izmerili smo tudi čas 30 nihajev, kjer meritve zopet predstavimo v tabeli
\begin{table}[H]
    \centering
    \begin{tabular}{|c|c|}
         Meritev 30 nihajev& t [s]  \\
         \hline
         1&56,40 \\
         2&56,17\\
         3&56,28\\
         4& 56,32\\
    \end{tabular}
    \caption{Tabela meritev časa 30 nihajev pri odklonu v isto smer}
    \label{tab:my_label}
\end{table}
Izračunamo povprečen čas 30 nihajev
$$ \bar{t} = \frac{t_1+t_2+t_3+t_4}{4} = 56,293 s \pm 0,005 s
$$
Absolutna napaka posamezne meritve je $\delta t = 0,01 s$, ki pa se s povprečenjem zmanjša kot $\delta \bar{t} = \frac{\delta t}{\sqrt{n}} = \frac{0,01 s}{\sqrt{4}} = 0,005 s$.\\

Povprečen nihajni čas enega nihaja je tako $$t_0 = \frac{\bar{t}}{30} = 1,876 s \pm 0,0002 s$$.

Izračunamo lahko tudi krožno frekvenco nihala kot $$\omega_0 = \frac{2\pi}{t_0} = 3,349 s^{-1}$$

\section*{Odklon nihal v nasprotni smeri}

Pri drugem delu smo merili iste količine, le da smo nihali odklonili v nasprotnih smereh. Dolžina je v tem primeru $d=17 cm $.\\
Podatki so predstavljeni v tabeli
\begin{table}[H]
    \centering
    \begin{tabular}{|c|c|}
    Meritev 30 nihajev&t [s]\\
    \hline
         1& 50,82 \\
         2& 50,46\\
         3&50,20\\
         4&50,66\\
    \end{tabular}
    \caption{Tabela meritev časa 30 nihajev pri odklonu v nasprotno smer}
    \label{tab:my_label}
\end{table}

Po istem postopku izračunamo povrpečen nihajni čas enega nihaja in krožno frekvenco nihala
$$ t_1 = 1,685 s \pm 0,0002 s 
$$
$$\omega_1 = 3,729 s^{-1}
$$
\section*{Odklonitev posameznega nihala}
Zdaj odklonimo le eno nihalo, drugo pa na začetku miruje v ravnovesni legi. Merili smo čas 15 nihajev vsakega nihala in zopet podatke zbrali v tabeli
\begin{table}[H]
    \centering
    \begin{tabular}{|c|c|c|}
         Meritev 15 nihajev& Nihalo 1 t [s]& Nihalo 2 t [s] \\
         \hline
         1&28,53&27,48\\
         2&28,65&27,66\\
         3&28,31&27,52\\
    \end{tabular}
    \caption{Meritev 15 nihajev posameznega nihala}
    \label{tab:my_label}
\end{table}
Po podobnih postopkih lahko izračunamo povprečni čas $t'$ in krožno frekvenco $\omega'$

$$t' =  1,8683 s
$$
$$\omega' = 3,363 s^{-1}
$$
Izmerili smo tudi čas med dvema mirovnima legama
$$ T = 18,5 s \pm 0,1 s
$$
ter frekvenco utripanja $$\omega_u = \frac{2\pi}{T} = 0,340 s^{-1}
$$
\chapter*{Izračuni}
\section*{Odklon nihal v isti smeri}
Izračunamo lahko koeficent prožnosti $D = m g d_0 =10,6 \frac{m^2 kg}{s^2} (1\pm 0,07) $.
Del naloge je tudi primerjava izmerjenih količin s teoretično izračunanimi. Za prvo nihanje lahko izračunamo nihajni čas in frekvenco kot
$$\Tilde{t_0} = 2\pi\sqrt{\frac{d_0}{g}} = 1,855 s (1\pm0,006)
$$
$$\Tilde{\omega_0} = \frac{2\pi}{\Tilde{t_0}} = 3,387 s^{-1}
$$
kjer z znakom $\Tilde{}$ označim izračunano vrednost.\\
Izračunamo lahko tudi vztrajnostni moment nihala kot 
$$ \Tilde{\omega_0} = \sqrt{\frac{D}{J}} \longrightarrow J = \frac{D}{\omega_0} =  3,13 \frac{m^2 kg}{s} (1\pm 0,013)
$$kjer sem uporabila izračunano vrednost.\\

\section*{Odklon nihal v nasprotni smeri}
Izračunajmo koeficient navora v tem primeru D' kot 
$$ D' = k d^2 = 0,6990 N m
$$
in frekvenco
$$\Tilde{\omega_1}=\sqrt{\frac{D+2D'}{J}}= 1,858 s^{-1} (1\pm0,026)
$$
$$\Tilde{t_1} = \frac{2\pi}{\Tilde{\omega_1}} = 1,69 s (1\pm 0,042)
$$

\section*{Odklonitev posameznega nihala}
Tudi pri tretjem načinu nihanja lahko izračunamo nihajne čase in frekvence kot
$$\Tilde{t'}=\frac{2\Tilde{t_0}\Tilde{t_1}}{\Tilde{t_0}+\Tilde{t_1}} = 1,769 s (1\pm 0,032)
$$
$$\Tilde{\omega'}= \frac{\Tilde{\omega_0}+\Tilde{\omega_1}  }{2}= 2,865 \ s^{-1} (1\pm 0,032)
$$
Izračunamo lahko tudi čas med dvema mirovnima legama
$$\Tilde{T} = \frac{t_1t_0}{t_0-t_1} = 19,00 s (1\pm 0,025)
$$
ter frekvenco utripanja 
$$\Tilde{\omega_u}= \Tilde{\omega_0} - \Tilde{\omega_1} = 0,36 s^{-1} (1\pm 0,022)
$$
\section*{Tabela rezultatov}
Vse rezultate, izmerjene in izračunane lahko predstavim v tabeli
\begin{table}[H]
    \centering
    \begin{tabular}{c|c|c|c|c|c|c|c|c}
         Količina& $t_0$ [s]& $t_1$[s]&$t'$ [s]& $\omega_0$ $s^{-1}$ & $\omega_1$ $s^{-1}$&$\omega'$ $s^{-1}$& T[s]&$\omega_u$ $s^{-1}$\\
         \hline
         Izmerjeno&1,876&1,685&1,8683&3,349&3,729&3,363&18,5&1,340\\
         Izračunano&1,855&1,69&1,769&3,387&1,858&2,865&19,0&0,36\\
    \end{tabular}
    \caption{Tabela vseh izmerjenih in izračunanih vrednosti}
    \label{tab:my_label}
\end{table}
Izračunamo lahko še faktorja sklopitve $K_1$ in $K_2$ 

$$ K_1 = \frac{1-\frac{\omega_0}{\omega_1}}{1+\frac{\omega_0}{\omega_1}} = 0,107
$$
$$K_2 = \frac{D'}{D+D'} = 0,0062 (1\pm 0,22)
$$
\end{document}

