\documentclass[a4paper]{report}
\author{Jure Kos}
\title{Vaja 40, Wheatstonov most}
\date{20.4.2022}
\usepackage{graphicx}
\graphicspath{ {./images/} }

\begin{document}

\maketitle



\chapter*{Uvod}
Zvezo med pritisnjeno napetostjo $U$ in električnim tokom $I$, ki teče skozi
prevodnik, opisuje Ohmov zakon. Faktor, ki povezuje napetost in tok se
imenuje upor $R$ in ima enoto $\Omega$ (Ohm).

\begin{equation}
R=\frac{U}{I}
\end{equation}

Upor je odvisen od dolžine $l$ in preseka S prevodnika ter od specifičnega
upora $\zeta$. Za upornike s konstantnim presekom velja enačba (2).\\

\begin{equation}
R=\frac{\zeta l}{S},
\end{equation}

Upor lahko zelo natančno merimo z Wheatstonovim mostom, na katerem
primerjamo napetosti v dveh tokovnih vejah električnega kroga. Slika 1
prikazuje shemo Wheatstonovega mostu uporabljenega pri tej vaji.


Veja AB je narejena iz enakomerno debele uporovne žice, s prilagajanjem
dolžine a pa lahko določimo položaj, ko v veji CD ni toka. V tem položaju velja enakost iz enačbe (3), pri čemer je $R$ znan upor, $l$ pa dolžina uporovne žice.\\

Iz enačbe $R_1$/$R_2$ = $R_3$/$R_4$ lahko izračunamo enega od uporov, če so drugi trije znani.\\

\begin{equation}
R_x = R_0 \frac{a}{l-a}
\end{equation}


\chapter*{Naloga}

Izmeriti upora danega upornika in žice. Izračunati specifični upor žice.

\section*{Potrebščine}
1. Ravnilo z merilno žico in drsnikom,\\
2. usmernik 2 V,\\
3. uporovna dekada,\\
4. ampermeter,\\
5. 8 žic z bananami,\\\
6. merjenca: upornik in žica.\\



\chapter*{Meritve}


  \begin{center}
   Dimenzije žice:\\
    \begin{tabular}{cc}
      dolžina  [cm] & $100 \pm 0,1$\\
      premer  [mm] & $0,05 \pm 0,05$\\
    \end{tabular}
  \end{center}



  \begin{center}
  Meritve neznanega upornika\\
    \begin{tabular}{|c|c|c|}
    \hline
      R [k$\Omega$] & a [cm] & L-a [cm]\\
      \hline
      1 & 83,3 & 17,1\\
      2 & 70,7 & 29,8\\
      3 & 61,0 & 38,9\\
      4 & 55,0 & 46,3\\
      5 & 49,1 & 51,8\\
      6 & 43,6 & 55,6\\
      7 & 40,6 & 59,9\\
      8 & 36,4 & 62,8\\
      9 & 34,0 & 65,5\\
      \hline
    \end{tabular}
  \end{center}



  \begin{center}
  Meritve upora žice\\
    \begin{tabular}{|c|c|c|}
    \hline
       R [$\Omega$] & a [cm] & L-a [cm]\\
      \hline
      1 & 85,9 & 14,9\\
      2 & 74,4 & 25,6\\
      3 & 66,5 & 33,6\\
      4 & 59,3 & 40,5\\
      5 & 53,8 & 46,0\\
      6 & 49,3 & 50,6\\
      7 & 45,4 & 54,1\\
      8 & 42,1 & 57,6\\
      9 & 39,4 & 60,3\\
      
      \hline    
    \end{tabular}
  \end{center}

\section*{Obdelava meritev}
Z uporabo enačbe 

\[R_x = R_0 \frac{a}{l-a}\]

lahko izračunamo vrednost upora $R_x$ za vsako meritev iz tabele 2. Iz meritev lahko izračunamo povprečje in ocenimo napako meritve. Dobljena
vrednost neznanega upora je 

\[R_x = 4750 \Omega\pm 0,1 \Omega\]

\noindent Enak postopek izvedemo še za meritve upora žice. Iz izračunanega povprečja in ocenjene napake je upor žice enak 

\[R_x = 5,9 \pm 0,1 \Omega \]

 Iz dobljenega rezultata lahko izračunamo še specifični upor žice, ki je enak
 
 \[\zeta = 4,6 \cdot 10^{-2}  \frac{\Omega \hspace{1mm} mm^2}{m} \pm 0,1  \frac{\Omega\hspace{1mm} mm^2}{m} \]


\end{document}