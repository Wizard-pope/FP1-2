\documentclass[a4paper]{article}

\title{Vaja 28, Specifična toplota trdne snovi}
\author{Jure Kos}
\date{4.11.2021}

\begin{document}

\maketitle

\section{Uvod}

Temperaturi dveh dotikajočih se in od okolice izoliranih teles s toplotnima kapacitetama $C_1$ in $C_2$ ter začetnima temperaturama $T_1$ in $T_2$ se po daljšem času izenačita. Telesi imata tedaj zmesno temperaturo $T_z$. Toplota, ki jo prvo telo odda, je enaka toploti, ki jo drugo telo prejme. Če med tem ne pride do fazne ali kemijske spremembe, lahko pri manjših temperaturnih razlikah računamo s sorazmernostjo, torej

\[C_1(T_1-T_z)=C_2(T_z-T_2)\]

Če poznamo toplotno kapaciteto $C_2$ enega od teles, lahko z izmerjenimi $T_1$,$T_2$,$T_z$ izračunamo toplotno kapaciteto $C_1$ drugega telesa:

\[C_1= \frac{T_z-T_2}{T_1-T_z}C_2\]

Pri homogenem telesu je C1 = mcp, kjer je m masa, cp pa specifična toplota snovi. Izračunamo jo iz prejšnje enačbe

\[c_p=\frac{T_z-T_2}{T_1-T_z}\frac{C_2}{m}\]


\section{Naloga}

Določiti specifično toploto 2 merjencev.



\section{Potrebščine}


1. Kalorimeter, \\
2. digitalni merilec temperature, \\
3. osebni računalnik, \\
4. tehtnica, \\
5. priprava za segrevanje merjenca, \\
6. merjenci. \\


\section{Potek}

Meritve te vrste opravimo po navadi z vodnim kalorimetrom. Ta je sestavljen iz kovinske kalorimetrske posode, ki stoji na plutovinastih zamaških (zaradi toplotne izolacije) v večji kovinski posodi, vse skupaj pa obdaja stiroporni plašč. Posodi pokrijemo s pokrovom, ki slabo prevaja toploto, skozenj pa vtaknemo v kalorimetrsko posodo še termometer. Plast zraka med posodama močno zmanjšuje toplotno izmenjavo med deli kalorimetra, stiroporni plašč pa preprečuje vplive okolice na meritev. V kalorimetrsko posodo potopimo še s plastiko prevlečeni trajni magnet, ki ga obračamo vrteče se magnetno polje v podstavku pod stiropornim plaščem (magnetno mešalo). Za meritev napolnimo kalorimetrsko posodo s stehtano količino vode. Posebej je treba določiti skupno toplotno kapaciteto kalorimetrske posode s termometrom, mešalom in vodo $C_2$.

\section{Meritve}
Izmerili smo mase vseh 3 merjencev, posode, posode z vodo in mešala. 


\begin{center}
\begin{tabular}{ |c|c| } 
 \hline
 $m_m$ & 725,22g $\pm{0,01}$g  \\ 
 $m_z$ & 679,67g $\pm{0,01}$g  \\ 
 $m_a$ & 230,56g $\pm{0,01}$g  \\ 
 $m_{pos}$ & 291,43g $\pm{0,01}$g  \\
 $m_{mes}$ & 8,58g $\pm{0,01}$g  \\
 $m_{p+v1}$ & 932,55g $\pm{0,01}$g  \\
 $m_{p+v2}$ & 909,75g $\pm{0,01}$g  \\
 $m_{p+v3}$ & 906,27g $\pm{0,01}$g  \\

 \hline
\end{tabular}
\end{center}

Poleg tega poznamo specifične toplote vode, medenine, železa in aluminija ter tudi specifično toploto termometra in mešala, saj sta iz železa.

\begin{center}
\begin{tabular}{ |c|c| } 
 \hline
 $c_v$ & 4200 J/kgK  \\ 
 $c_z$ & 494 J/kgK  \\ 
 $c_a$ & 900 J/kgK  \\ 
 $c_m$ & 360 J/kgK  \\ 
 \hline
\end{tabular}
\end{center}

Po 3 meritvah smo pridobili naslednje podatke:

1. Meritev (aluminij):
\begin{center}
\begin{tabular}{ |c|c| } 
 \hline
 $T_m1$ & 367K  $\pm{1}$K \\ 
 $T_v1$ & 296,7K $\pm{0,1}$K\\ 
 $T_z1$ & 301,6K $\pm{0,1}$K \\ 
 \hline
\end{tabular}
\end{center}

2. Meritev (železo):
\begin{center}
\begin{tabular}{ |c|c| } 
 \hline
 $T_m2$ & 365K  $\pm{1}$K \\ 
 $T_v2$ & 296,5K $\pm{0,1}$K\\ 
 $T_z2$ & 303,7K $\pm{0,1}$K \\ 
 \hline
\end{tabular}
\end{center}

3. Meritev (medenina):
\begin{center}
\begin{tabular}{ |c|c| } 
 \hline
 $T_m3$ & 364K  $\pm{1}$K \\ 
 $T_v3$ & 293.1K $\pm{0,1}$K\\ 
 $T_z3$ & 300,1K $\pm{0,1}$K \\ 
 \hline
\end{tabular}
\end{center}

\section{Računi}

Iz izmerjenih dimenzij termometra, gostote železa in specifične toplote železa lahko izračunamo toplotno kapaciteto termometra. ($\rho_z=7,87g/cm^3$)

\[C_t=\left(\frac{d}{2}\right)^2\pi l\rho_z c_z=\left(\frac{0,400\cdot(1\pm0,025)cm}{2}\right)^2\pi\cdot10,85 \cdot(1\pm0,002)cm\cdot 7,87\frac{g}{cm^3}\cdot 0,494 \frac{J}{gK}\] \\
\[\approx5,3 \cdot (1 \pm0,052)\frac{J}{K}\]

Iz mase mešala in njegove specifične toplote lahko ponovimo postopek:

\[C_m=m_m c_z=5,58\cdot(1\pm0,002)g\cdot0,494 \frac{J}{gK} \approx2,8\cdot(1\pm0,002)\frac{J}{K}\]

Sedaj lahko izračunamo skupno toplotno kapaciteto posode z vodo, mešalcem in termometrom za 1. meritev. ($c_m=c_{pos}$)

\[C_{2_a}=m_{pos}\cdot c_{pos}+m_{v1}\cdot c_v+C_t+C_m=\] \\
\[=\left((291,43g\pm0,01g)\cdot 0,360 \frac{J}{gK}\right)+\left((641,12g\pm0,01g)\cdot4,200\frac{J}{gK}\right)\]\\
\[+\left(5,3\frac{J}{K}\pm0,28\frac{J}{K}\right)+\left(2,8\frac{J}{K}\pm0,006\frac{J}{K}\right)\]\\
\[\approx2805,7\frac{J}{K}\pm0,306\frac{J}{K}\]

Za 2. in 3. meritev tako dobimo

\[C_{2_z}\approx2710,0\frac{J}{K}\pm0,306\frac{J}{K}\]

\[C_{2_m}\approx2695,3\frac{J}{K}\pm0,306\frac{J}{K}\]
\pagebreak

Sedaj lahko izračunamo specifično toploto merjencev.

\[c_a=\frac{T_{z1}-T_{v1}}{T_{m1}-T_{z1}}\cdot\frac{C_2}{m_a}=\frac{301,6K\pm0,1K-296,7K\pm0,1K}{367K\pm1,0K-301,6K\pm0,1K}\cdot\frac{2805,7J\pm0,306J}{0,23056kg\pm0,00001kg\cdot K}=\]\\
\[=\frac{4,9K\pm0,2K}{65,4K\pm1,1K}\cdot\frac{2805,7J\pm0,306J}{0,23056kg\pm0,00001kg\cdot K}=\]\\
\[=\frac{4,9\cdot(1\pm0,041)K}{65,4\cdot(1\pm0,017)K}\cdot\frac{2805,7\cdot(1\pm0,0001)J}{0,23056\cdot(1\pm0,00004)kg\cdot K}=\]\\
\[\approx 911,7\frac{J}{kgK}\pm53,0\frac{J}{kgK}\]

Enako izračunamo za železo in medenino.

\[c_z\approx 468,3\frac{J}{kgK}\pm21,6\frac{J}{kgK}\]

\[c_m\approx 407,1\frac{J}{kgK}\pm18,8\frac{J}{kgK}\]

\section{Analiza}
Edina izmerjena vrednost, ki pade v območje napake, je specifična toplota aluminija. Kar pomeni, da so bile napake pri merjenju napačno ocenjene ali pa se je zgodila sistemska napaka, kar je privedlo do nepravilnih rezultatov.

\end{document}
